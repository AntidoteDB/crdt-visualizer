\documentclass{scrreprt}
\usepackage{listings}
\usepackage{underscore}
\usepackage[bookmarks=true]{hyperref}
\hypersetup{
	bookmarks=false,    % show bookmarks bar?
	pdftitle={Software Requirement Specification},    % title
	pdfauthor={Yiannis Lazarides},                     % author
	pdfsubject={TeX and LaTeX},                        % subject of the document
	pdfkeywords={TeX, LaTeX, graphics, images}, % list of keywords
	colorlinks=true,       % false: boxed links; true: colored links
	linkcolor=blue,       % color of internal links
	citecolor=black,       % color of links to bibliography
	filecolor=black,        % color of file links
	urlcolor=purple,        % color of external links
	linktoc=page            % only page is linked
}%
\def\myversion{1.0 }
\title{%
	\flushleft
	\rule{16cm}{5pt}\vskip1cm
	\Huge{SOFTWARE REQUIREMENTS\\ SPECIFICATION}\\
	\vspace{2cm}
	for\\
	\vspace{2cm}
	An interactive tool for demonstrating CRDTs\\
	\vspace{2cm}
	\LARGE{Release 1.0\\}
	\vspace{2cm}
	Prepared by:\\
	Yassin Frekha \\
   	Rasha Abu Qasem\\
	\vfill
	\rule{16cm}{5pt}
}
\date{}
\usepackage{hyperref}
\begin{document}
	\maketitle
	\tableofcontents
	\chapter{Introduction}
	
	\chapter{Requirements List}
	\section{Functional Requirements}
	\begin{itemize}
	\item The number of replicas should be a variable specified in the markdown document.
	\item The visualizer could be embedded anywhere in the document.
	\item Several instances of the visualizer can run simultaneously in the same document. 
	\item An initial example for each data type should be specified in the markdown document.
	\item The user can add an operation of the data type by clicking on a point on the replica.
	\item After clicking on a point a text-field would be displayed for the user.
	\item The user can add the desired operation by writing in a text-field.
	\item The text-field should have auto-complete feature.
	\item Only appropriate operations for a data-type are allowed.  
	\item The user can remove an operation by dragging the operation point and dropping it in the trash(needs investigation). 
	\item The user can update the states between replicas by drawing an arrow from the source replica to the designated replica.
	\item The state of the replica should be updated according to the execution plan.
	
	\end{itemize}
	\section{Non-Functional Requirements}
	\begin{itemize}
	\item The project will be implemented with Typescript.
	\item The design should be extendable.
	\item Priorities of implementation:
	\begin{enumerate}
	\item Number Type
	\item Sets
	\item Flags
	\item LWW-Register
	\item MV-Register
	\item Maps
	\end{enumerate}
	\end{itemize}
	
\end{document}